\begin{footnotesize}
%\begin{small}
\textbf{General Usage} \\
\begin{tabular}{p{3.6 in} p{3.0 in}}
\texttt{hashdb <command> <options> <args>} & Run hashdb command \\
\end{tabular}
\\
\\
\textbf{New Database} \\
\begin{tabular}{p{3.6 in} p{3.0 in}}
\hangindent=2em \texttt{create [-b <block size>] [-a <byte alignment>] [-m <max source, offset pairs>] [-t <hash prefix bits:hash suffix bytes>] <hashdb.hdb>} &
Create a new hash database \\
\end{tabular}
\\
\\
\textbf{Import/Export} \\
\begin{tabular}{p{3.6 in} p{3.0 in}}
\hangindent=2em\texttt{ingest [-r <repository name>] [-w <whitelist.hdb>] [-s <step size>] [-x <rel>] <hashdb.hdb> <import directory>} &
Import from path recursively into hash database, labeling hashes in the whitelist and hashes matching entropy traits, can disable recursion, entropy, labels \\
\hangindent=2em\texttt{import\_tab [-r <repository name>] [-w <whitelist.hdb>] <hashdb.hdb> <tab.txt>} &
Import from tab file into hash database, labeling hashes in the whitelist\\
\hangindent=2em\texttt{import <hashdb.hdb> <hashdb.json>} &
Import JSON format data into hash database \\
\hangindent=2em\texttt{export <hashdb.hdb> <hashdb.json>} &
Export hash database data in JSON format \\
\end{tabular}
\\
\\
\textbf{Database Manipulation} \\
\begin{tabular}{p{3.6 in} p{3.0 in}}
\texttt{add <A.hdb> <B.hdb>} & $A \rightarrow B$ add $A$ into $B$ \\
\texttt{add\_multiple <A.hdb> <B.hdb> ... <C.hdb>} & $A + B + \ldots \rightarrow C$ add $A$, $B$, \ldots into $C$\\
\hangindent=2em\texttt{add\_repository <A.hdb> <B.hdb> <repository name>} & \hangindent=2em$A_r \rightarrow B$ add when repository name matches \\
\texttt{add\_range<A.hdb> <B.hdb> <m:n>} & $A_{m:n} \rightarrow B$ add hashes that have source counts within range, inclusive\\
\texttt{intersect <A.hdb> <B.hdb> <C.hdb>} & $A \cap B \rightarrow C$ add when hash and source are common\\
\texttt{intersect\_hash <A.hdb> <B.hdb> <C.hdb>} & $A \cap B \rightarrow C$ add when hashes are common\\
\texttt{subtract <A.hdb> <B.hdb> <C.hdb>} & $A - B \rightarrow C$ add when hash and source not common\\
\texttt{subtract\_hash <A.hdb> <B.hdb> <C.hdb>} & $A - B \rightarrow C$ add when hashes are not common\\
\hangindent=2em\texttt{subtract\_repository <A.hdb> <B.hdb> <repository name>} & \hangindent=2em$A_{\overline{r}} \rightarrow B$ add unless repository name matches\\
\end{tabular}
\\
\\
\textbf{Scan} \\
\begin{tabular}{p{3.6 in} p{3.0 in}}
\texttt{scan\_list <hashdb.hdb> <hashes file>} & Scan hashes file for hash match \\
\texttt{scan\_hash <hashdb.hdb> <hex block hash>} & Scan for hash match \\
\hangindent=2em\texttt{scan\_image [-s <step size>] [-x <r>] <hashdb.hdb> <media image file>} & Scan media image for hash match, can disable recursion \\
\end{tabular}
\\
\\
\textbf{Statistics}\\
\begin{tabular}{p{3.6 in} p{3.0 in}}
\texttt{size <hashdb.hdb>} & Print size information for internal database tables \\
\texttt{sources <hashdb.hdb>} & Print source information \\
\texttt{histogram <hashdb.hdb>} & Print hash distribution \\
\texttt{duplicates <hashdb.hdb> <number>} & Print hashes sourced the given number of times \\
\texttt{hash\_table <hashdb.hdb> <hex file hash>} & Print hashes associated with the source file hash\\
\texttt{read\_bytes <image file> <offset> <count>} & Print raw bytes from the image file\\
\end{tabular}
\\
\\
\textbf{Performance Analysis}\\
\begin{tabular}{p{3.6 in} p{4 in}}
\hangindent=2em\texttt{add\_random <hashdb.hdb> <count>} & Add random hashes, log to \texttt{timestamp.json}\\
\texttt{scan\_random <hashdb.hdb> <count>} & Scan random hashes, log to \texttt{timestamp.json}\\
\hangindent=2em\texttt{add\_same <hashdb.hdb> <count>} & Add same hashes, log to \texttt{timestamp.json}\\
\texttt{scan\_same <hashdb.hdb> <count>} & Scan same hashes, log to \texttt{timestamp.json}\\
\end{tabular}
\\
\\
\textbf{\bulk Scanner}\\
%\begin{tabular}{p{5.8 in} l}
\begin{tabular}{p{5.6 in} p{2 in}}
\texttt{bulk\_extractor -E hashdb -S hashdb\_mode=import -o outdir1 -R my\_import\_dir} & Import directory\\
\texttt{bulk\_extractor -E hashdb -S hashdb\_mode=import -o outdir1 my\_media\_image} & Import image\\
\hangindent=2em\texttt{bulk\_extractor -E hashdb -S hashdb\_mode=scan -S hashdb\_scan\_path= outdir1/hashdb.hdb -o outdir2 my\_image2} & Scan image\\
\end{tabular}
\end{footnotesize}
%\end{small}

