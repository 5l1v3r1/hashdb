
%\documentclass[10pt,twoside,twocolumn]{article}
\documentclass[12pt,twoside]{article}
\usepackage[bf,small]{caption}
\usepackage[letterpaper,hmargin=1in,vmargin=1in]{geometry}
\usepackage{paralist} % comapctitem, compactdesc, compactenum
\usepackage{titlesec}
\usepackage{titletoc}
\usepackage{times}
\usepackage{hyperref}
\usepackage{algorithmic}
\usepackage{graphicx}
\graphicspath{{./graphics/}}
\usepackage{xspace}
\usepackage{verbatim}
\hyphenation{Sub-Bytes Shift-Rows Mix-Col-umns Add-Round-Key}

\setlength{\parskip}{12pt}
\setlength{\parindent}{0pt}

\newcommand{\hdb}{\emph{hashdb}\xspace}
\newcommand{\bulk}{\emph{bulk\_extractor}\xspace}
\newcommand{\mdd}{\emph{md5deep}\xspace}
\newcommand{\mi}{\textbf{map iterator}\xspace}
\newcommand{\mmi}{\textbf{multimap iterator}\xspace}

\begin{document}

\title{\hdb Design Document}
\author{Bruce Allen}
\maketitle

\begin{abstract}
This document defines the design of the \hdb tool and library
for \hdb v1.0.0, which is different from that of \hdb v0.9.1.
The \hdb Checker and Manager are merged.
the \hdb Library interfaces are changed and support more capability.
\end{abstract}

\section{Changes for v1.0.0}
\paragraph{Database Types}
Two parameters define the database type when creating a new database:
\begin{compactitem}
\item \textbf{Data Size} specifies the block size,
or select 0 to store hashes of any size, useful for managing file hashes.
\item \textbf{Hash Type} specifies the hash type, such as MD5 or SHA1.
We do not mix hash types in a database because hash types have differing sizes.
If mixing hash sizes is really desired, we can truncate or zero-fill
hash values in the database.
\end{compactitem}
\paragraph{Remote Modify}
We can add or remove hash elements remotely over a socket.
\paragraph{Renamed Commands}
Some \hdb tool commands are renamed to specifically indicate their function.

\section{\hdb Services}
\hdb services are provided by the \hdb tool and the \hdb library.
\subsection{\hdb Tool Commands}
\begin{compactitem}
\item Create a \hdb database
\item Import hashes from DFXML
\item Export hashes to DFXML
\item Add hashes in DFXML into database
\item Copy a database to a second database
\item Merge two databases into a third database
\item Remove hashes in DFXML from database
\item Rebuild Bloom filters for database
\item Get hash source information
\item Get database information
\item Run as a Server service
\end{compactitem}

Specifically:
\begin{compactitem}
\item \texttt{create <hashdb settings> <bloom filter settings> <hashdb>} \\
Create a new empty \hdb given the specified settings.
\item \texttt{import [-r <repository name>] <DFXML file> <hashdb>} \\
Import hashes from the DFXML file into the existing \hdb
using the given repository name.
\item \texttt{export <hashdb> <DFXML file>} \\
Export hashes from the hashdb to the new DFXML file.
\item \texttt{copy <hashdb 1> <hashdb 2>} \\
Copy hashes in database 1 into database 2
using a \mmi over database 1.
\item \texttt{merge <hashdb 1> <hashdb 2> <hashdb 3>} \\
Merge databases 1 and 2 into 3
using {\mmi}s over databases 1 and 2 in sorted order.
\item \texttt{remove <hashdb 1> <hashdb 2>} \\
Remove hashes in database 1 from database 2
using a \mmi over database 1.
\item \texttt{remove\_dfxml <DFXML file> <hashdb>} \\
Remove hashes in the DFXML file from the database.
\item \texttt{deduplicate <hashdb>} \\
Removes all duplicates.
\item \texttt{rebuild\_bloom <bloom filter settings> <hashdb>} \\
Rebuild the Bloom filters with new settings
using a \mi over the given database.
\item \texttt{server <hashdb> <socket>} \\
Start the \hdb tool as a server service using the database
at the specified socket.
This service uses all \hdb Library interfaces.
\item \texttt{query\_hash <path or socket> <DFXML file>} \\
Print out hashes in the \hdb that match hashes in the DFXML file.
\item \texttt{get\_hash\_source <path or socket> <identified\_blocks.txt> <output text file>} \\
Create output file from input, showing full source intormation.
\item \texttt{get\_hashdb\_info <path or socket>} \\
Print out information about the given database.
\end{compactitem}

\subsubsection{Changes from \hdb v0.9.1}
\begin{compactitem}
\item The parameters for the tuning parameters and bloom filters
remain the same.
\item The \texttt{create} command is new.
The user must create a database before using \texttt{add} or \texttt{merge}.
Alternatively, \texttt{merge} could require parameters for creating
a new database.
\item The \hdb tool uses the \hdb library
for services the \hdb library provides.
\end{compactitem}

\subsection{\hdb Library Interfaces}
\begin{compactitem}
\item Open
\item Insert hashes
\item Remove hashes
\item Find hashes
\item Look up hash sources
\item Get \hdb information
\end{compactitem}

These interfaces permit future expansion to other hash types
by having the hash type in their name.
To start, the \emph{MD5} hash type is supported.

% open
\subsubsection{Open \hdb}
Open a \hdb service that uses the database at the given file path or socket.
If the string contains adjacent // markers, a socket is indicated,
otherwise a file path is indicated.
Here is the constructor:
\begin{small}
\begin{verbatim}
hashdb_md5_h(const std::string &hashdb_service);
\end{verbatim}
\end{small}

% insert hashes
\subsubsection{Insert Hashes}
Insert the hashes and their associated source information
into the opened database and log the change to file \texttt{log}.
\begin{small}
\begin{verbatim}
insert_md5_hashes(const insert_md5_hashes_t &insert_hashes);

struct insert_md5_hash_t {
    uint8_t digest[16];
    std::string repository_name;
    std::string filename;
    uint64_t file_offset;
};

typedef std::vector<insert_md5_hash_t> insert_md5_hashes_t;
\end{verbatim}
\end{small}

% remove hashes
\subsubsection{Remove Hashes}
Removes hashes that exactly match the specified source information
and log the change to file \texttt{log}.
The data structure required for erasing hashes
is the same as that for inserting hashes.

\begin{small}
\begin{verbatim}
remove_md5_hashes(const remove_md5_hashes_t& remove_hashes);

typedef insert_md5_hash_t remove_md5_hash_t;

typedef insert_md5_hashes_t remove_md5_hashes_t;
\end{verbatim}
\end{small}

% find hashes
\subsubsection{Find Hashes}
Finds matching hashes and returns the hashes that match,
along with the \emph{source lookup encoding}
associated with matching hashes.
A hash index is used so that the callee can map responses back
to their original request entries.

\begin{small}
\begin{verbatim}
find_md5_hashes(const find_md5_hashes_t &find_hashes,
            found_md5_hashes_t &found_hashes);

struct find_md5_hash_t {
    uint32_t id;
    uint8_t digest[16];
};

typedef std::vector<find_md5_hash_t> find_md5_hashes_t;

struct found_md5_hash_t {
    uint32_t id;
    uint8_t digest[16];
    uint32_t count;
};

typedef std::vector<found_md5_hash_t> found_md5_hashes_t;
\end{verbatim}
\end{small}

% look up hash sources
\subsubsection{Look Up Hash Sources}
Look up source information for hash values provided.
There may be multiple sources for a hash value.
Source information consists of the repository name, filename, and file offset
of where the hash was sourced from.

\begin{small}
\begin{verbatim}
look_up_md5_hash_sources(const look_up_md5_hash_sources_t& look_up_hash_sources,
                         looked_up_md5_hash_sources_t& looked_up_hash_sources);

struct look_up_md5_hash_source_t {
    uint8_t digest[16];
};

typedef std::vector<look_up_md5_hash_source_t> look_up_md5_hash_sources_t;

struct looked_up_md5_hash_source_t {
    std::string repository_name;
    std::string filename;
    uint64_t file_offset;
};

typedef std::vector<pair<uint8_t[16], std::vector<looked_up_md5_hash_source_t>
                looked_up_md5_hash_sources_t;
\end{verbatim}
\end{small}

% get hashdb info
\subsubsection{Get \hdb Information}
Returns information about the opened database.

\begin{small}
\begin{verbatim}
get_hashdb_info(std::string& information);
\end{verbatim}
\end{small}

\subsection{\hdb Server}
The \hdb server services requests issued to the \hdb library.
\hdb server actions are straightforward.

When a \hdb library service or the \hdb tools need to use a \hdb server,
and the service is at a local path rather than at a socket,
it should perform the following steps:
\begin{algorithmic}
\STATE create the server service as a thread using the machine-local socket
\STATE use the server service using the system-local socket
\STATE destroy the locally running server service thread
\end{algorithmic}

Here is the basic flow for the \hdb server:
\begin{algorithmic}
\STATE Instantiate a \hdb element manager
\WHILE{true}
  \IF{add hashes}
    \STATE create the change logger
    \WHILE{hashes to add}
      \STATE add hash using the \hdb element manager
    \ENDWHILE
  \ELSIF{remove hashes}
    \STATE create the change logger
    \WHILE{hashes to remove}
      \STATE remove hash using the \hdb element manager
    \ENDWHILE
  \ELSIF{find hashes}
    \WHILE{hashes to look up}
      \IF{match}
        \STATE add hash and count to response
      \ENDIF
    \ENDWHILE
  \ELSIF{get info}
    \STATE put info into response
  \ENDIF
\ENDWHILE
\end{algorithmic}

\section{\hdb Modules}
\subsection{\hdb Manager}
The \texttt{hashdb\_manager\_t}
provides services required for accessing the hash database
as a whole.  It hides the map manager, multimap manager, and bloom filter.
Changes to the \hdb are noted in the \hdb change logger.

Interfaces are:
\begin{compactitem}
\item Open a hash database as new, RW, or RO
\item Insert a \texttt{T, uint64\_t} entry noting the change.
\item Remove a \texttt{T, uint64\_t} entry noting the change.
\item Remove all elements of \texttt{<T>}, noting the change.
\item See if a hash of type \texttt{<T>} exists.
\item Provide a \hdb iterator via begin() and end()
for iterating over all hash values.
Iterator dereferences to \texttt{pair<T, uint64\_t>>}.
\end{compactitem}

% instantiation
\subsubsection{Open a \hdb Database}
Open a \hdb database at the specified directory
with a file mode of new, read-only, or read-write.
Operations may be performed with this database as allowed by its mode.

\begin{small}
\begin{verbatim}
hashdb_manager_t(const std::string &hashdb_dir, file_mode_t file_mode_type);
\end{verbatim}
\end{small}

% insert hash
\subsubsection{Insert Hash}
Insert the specified hash, noting the change.

\begin{small}
\begin{verbatim}
insert(const <T> &key, uint64_t pay,
       hashdb_change_object_t &hashdb_change_object);
\end{verbatim}
\end{small}

% remove hash
\subsubsection{Remove Hash}
Remove the specified hash, noting the change.

\begin{small}
\begin{verbatim}
remove(const <T> &key, uint64_t pay,
      hashdb_change_object_t &hashdb_change_object);

remove(const <T> &key,
      hashdb_change_object_t &hashdb_change_object);
\end{verbatim}
\end{small}

% has
\subsubsection{Has Hash}
Check for hash, checking Bloom filter first.

\begin{small}
\begin{verbatim}
has(const <T> &key);
\end{verbatim}
\end{small}

% size
% no, provide a complete stat report elsewhere.
%\subsubsection{Count}
%Returns the number of elements containing the given hash value.
%
%\begin{small}
%\begin{verbatim}
%uint32_t count(md5_t md5);
%\end{verbatim}
%\end{small}

% equal range
% no, not at this layer
%\subsubsection{Equal Range}
%Returns pair of \mmi to lower bound
%(first element with key not less than value)
%and upper bound (first element with key greater than value or end).
%
%\begin{small}
%\begin{verbatim}
%pair<const_hashdb_multimap_iterator lower,
%    const_hashdb_multimap_iterator upper> equal_range(md5_t md5);
%\end{verbatim}
%\end{small}

% mi begin
\subsubsection{\mi Begin}
Returns begin for \hdb iterator.

\begin{small}
\begin{verbatim}
hashdb_iterator_t<T> begin();
\end{verbatim}
\end{small}

% mi end
\subsubsection{\mi End}
Returns end for \hdb iterator.

\begin{small}
\begin{verbatim}
hashdb_iterator_t<T> end();
\end{verbatim}
\end{small}

\subsection{\hdb Iterator}
Provide an iterator over all \hdb elements, not just map.

\begin{small}
\begin{verbatim}
hashdb_iterator<T>(map_manager_t, multimap_manager_t);
\end{verbatim}
\end{small}

% \hdb element
\subsection{\hdb Element}
A \hdb element consists of a hash value and where it was sourced from.
Specifically:
\begin{small}
\begin{verbatim}
struct hashdb_element_t {
    uint8_t digest[SIZE];
    std::string repository_name;
    std::string filename;
    uint64_t file_offset;
};
\end{verbatim}
\end{small}

\subsection{\hdb Change Logger}
This structure is used for tracking changes to the \hdb.
The \hdb element manager increments variables
as requests are made to insert or remove \hdb elements.
Changes are written to file \texttt{log}
when the change logger instance is destroyed.
Additional interfaces permit additional logging,
specifically, timestamp and settings.

\begin{small}
\begin{verbatim}
class hashdb_change_logger_t {
      // insert
      uint32_t hashes_inserted;
      uint32_t hashes_not_inserted_invalid_file_offset;
      uint32_t hashes_not_inserted_wrong_hash_block_size;
      uint32_t hashes_not_inserted_wrong_hashdigest_type;
      uint32_t hashes_not_inserted_exceeds_max_duplicates;
      uint32_t hashes_not_inserted_duplicate_source;

      // remove
      uint32_t hashes_removed;
      uint32_t hashes_not_removed_invalid_file_offset;
      uint32_t hashes_not_removed_wrong_hash_block_size;
      uint32_t hashes_not_removed_wrong_hashdigest_type;
      uint32_t hashes_not_removed_no_hash;
      uint32_t hashes_not_removed_different_source;

      // constructor
      hashdb_change_logger_t(const std::string &hashdb_dir,
                             const std::string title);

      // log timestamp
      void add_timestamp(std::string text);

      // log hashdb_settings
      void add_hashdb_settings(const hashdb_settings_t &hashdb_settings);

      // hashdb_db_manager state
//      void add_hashdb_db_manager_state(const hashdb_db_manager_t &manager);
      void add_hashdb_db_manager_state(); // TBD

      // bloom state, this may be temporary
      void add_bloom_state(const bloom_filter_t &bloom_filter, int index);
};
\end{verbatim}
\end{small}

Note that logs are now written at the \hdb library level
for each invocation of the library,
rather than at the tool level, which logged once
for the invocation of the \hdb tool.
This is different from v0.9.1.

% source lookup manager
\subsection{Source Lookup Manager}
This module manages source lookups.
\begin{compactitem}
\item It wraps source lookup capability under one purposed interface.
\item It manages the lookup store, filename store,
and repository name store files required by the three B-Tree multisets,
and uses these storage files: \\
\texttt{
source\_lookup\_store.dat \\
source\_lookup\_store.idx1 \\
source\_lookup\_store.idx2 \\
source\_filename\_store.dat \\
source\_filename\_store.idx1 \\
source\_filename\_store.idx2 \\
source\_repository\_name\_store.dat \\
source\_repository\_name\_store.idx1 \\
source\_repository\_name\_store.idx2
}
\item Provides Interfaces to get source information
from \texttt{source\_lookup\_index},
to get the \texttt{source\_lookup\_index} from source information,
and to insert elements by specifying the repository name, filename,
and \texttt{source\_lookup\_index}.
\end{compactitem}

Source lookup manager interfaces follow.
% constructor
\subsubsection{Constructor}
Open the source lookup manager with file mode of new, RO, or RW.
\begin{small}
\begin{verbatim}
source_lookup_manager_t(const std::string p_hashdb_dir,
                        file_mode_type_t file_mode);
\end{verbatim}
\end{small}

% get source location
\subsubsection{Get Source Location}
Return the repository name and filename from the source lookup index.
\begin{small}
\begin{verbatim}
bool get_source_location(uint64_t source_lookup_index,
                         std::string &repository_name,
                         std::string &filename);
\end{verbatim}
\end{small}

% get source lookup index
\subsubsection{Get Source Lookup Index}
Return the source lookup index from the repository name and filename.
\begin{small}
\begin{verbatim}
bool get_source_lookup_index(const std::string &repository_name,
                             const std::string &filename,
                             uint64_t source_lookup_index);
\end{verbatim}
\end{small}

% insert source lookup element
\subsubsection{Insert Source Lookup Element}
Add a source lookup element, specifically, a repository name and filename,
and get back the new (or already existing) source lookup index.
\begin{small}
\begin{verbatim}
bool insert_source_lookup_element(const std::string &repository_name,
                                  const std::string &filename,
                                  uint64_t &source_lookup_index);
\end{verbatim}
\end{small}

\subsection{Bloom Filter}
\hdb uses two Bloom filters.
The Bloom filter module manages one Bloom filter instance.

% instantiation
\subsubsection{Instantiation}
Create one Bloom filter.
\begin{small}
\begin{verbatim}
bloom_filter_t(const std::string &filename,
               file_mode_type_t file_mode,
               const bloom_settings_t bloom_settings);
\end{verbatim}
\end{small}

% add hash value
\subsubsection{Add Hash Value}
Add one hash value to the Bloom filter.
\begin{small}
\begin{verbatim}
void add_hash_value(const uint8_t[SIZE] &hash);
\end{verbatim}
\end{small}

% is positive
\subsubsection{Is Positive}
Return true if the lookup is positive, indicating the hash value may be present.
\begin{small}
\begin{verbatim}
is_positive(const uint8_t[SIZE] &hash);
\end{verbatim}
\end{small}


\subsection{\hdb Settings}
The \hdb element manager and
many of the storage types in the \hdb database use configurable \hdb settings.
Refer to \texttt{hashdb\_settings}, \texttt{hashdb\_settings\_reader},
\texttt{hashdb\_settings\_writer}. 

% instantiation
\subsubsection{Instantiation}
Create a \hdb settings object:

\begin{small}
\begin{verbatim}
// new
hashdb_settings_t();

// from file
hashdb_settings_t(std::string hashdb_dir);

// make a copy
hashdb_settings_t(hashdb_settings_t);
\end{verbatim}
\end{small}

% instantiation
\subsubsection{Save Settings}
\begin{small}
\begin{verbatim}
save_settings(std::string hashdb_dir);
\end{verbatim}
\end{small}


\section{Low-Level \hdb Modules}

% map manager
\subsection{Map Manager}
This module manages mapped hash storage.
It has been renamed from \textit{Hash Store}.
\begin{compactitem}
\item The key \texttt{T} is the \texttt{uint8\_t[SIZE]} hash digest,
such as T,
and the value is the \texttt{uint64\_t} source lookup encoding.
\item The source lookup encoding consists of $32-40$ bits
of source lookup index and $32-24$ bits of block offset value,
but this detail does not matter at this low storage level.
\item The hash map maps to the \texttt{hash\_map\_store} file.
\item Provides Interfaces to insert, erase, look up, and change elements.
\item A \texttt{const\_iterator} is available for walking the hash map.
\end{compactitem}

Here are the interfaces:
\begin{small}
\begin{verbatim}
pair<map_iterator_t<T>, bool> emplace(T, uint64_t);
size_t erase(T);
pair<map_iterator_t<T>, bool> change(T, uint64_t);
map_iterator_t<T> find(T);
bool has(T);   // faster than find()
map_iterator_t<T> begin();
map_iterator_t<T> end();
size_t size();
\end{verbatim}
\end{small}

% multimap manager
\subsection{Multimap Manager}
This module manages multimap hash storage.
It has been renamed from \textit{Hash duplicates Store}.
\begin{compactitem}
\item The key \texttt{T} is the \texttt{uint8\_t[SIZE]} hash digest,
such as md5,
and the value is the \texttt{uint64\_t} source lookup encoding.
\item The source lookup encoding consists of $32-40$ bits
of source lookup index and $32-24$ bits of block offset value,
but this detail does not matter at this low storage level.
\item The hash multimap maps to the \texttt{hash\_multimap\_store} file.
\item Provides Interfaces to insert, erase, and look up elements.
\item A \texttt{const\_iterator} is available for walking
between elements \texttt{[a,b)} of a key in the hash multimap.
\item Begin and end accessors are not provided.
\end{compactitem}

Here are the interfaces:
\begin{small}
\begin{verbatim}
bool emplace(md5, uint64_t);
bool erase(md5, uint64_t);
size_t erase_range(md5);
pair<multimap_iterator_t<T>, multimap_iterator_t<T>> equal_range(T);
bool has(md5, uint64_t);
bool has_range(md5);
size_t size();
\end{verbatim}
\end{small}

% source lookup encoding
\subsection{Source Lookup Encoding}
The source lookup encoding provides algorithms for working with a
\texttt{uint64\_t}
as a source lookup encoding containing a source lookup index,
a hash block offset, and a count value.
These algorithms are
implemented in namespace \texttt{source\_lookup\_encoding}.
This encoding is an optimization for saving space in memory.
Here are some properties of this encoding:
\begin{compactitem}
\item If the embedded count value $=$ 1,
the source lookup index and hash block offset
have valid source lookup values.
The \texttt{uint64\_t} value is distributed between the
source lookup index and hash block offset as follows:
  \begin{compactitem}
  \item number of index bits goes into the source lookup index.
  \item 32 $-$ number of index bits goes into the hash block offset.
  \end{compactitem}
\item If the embedded count value $>$ 1,
the source lookup index and hash block offset accessors
are not used and return 0.
\item If the hash is a file hash, the block offset will be 0, as expected.
\item If an attempt is made to compose a source lookup encoding
from a source lookup index or a hash block offset
that is too large, an error will be emitted to \texttt{stderr}
and 0 will be returned.
\item A count value is indicated when the most significant 32 bits
of the \texttt{uint64\_t} $=$ \texttt{0xffff}.
\end{compactitem}

% accessors
\subsubsection{Accessors}
\begin{small}
\begin{verbatim}
static uint64_t get_source_lookup_encoding(
                 uint8_t source_lookup_index_bits,
                 uint64_t source_lookup_index,
                 uint64_t hash_block_offset);

static uint64_t get_source_lookup_encoding(uint32_t count);

static uint32_t get_count(uint64_t source_lookup_encoding);

static uint64_t get_source_lookup_index(
                 uint8_t source_lookup_index_bits,
                 uint64_t source_lookup_encoding);

static uint64_t get_hash_block_offset(
                 uint8_t source_lookup_index_bits,
                 uint64_t source_lookup_encoding);
\end{verbatim}
\end{small}

\end{document}

