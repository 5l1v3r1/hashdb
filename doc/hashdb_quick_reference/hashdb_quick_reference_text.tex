\begin{small}
\begin{footnotesize}
\textbf{General Usage} \\
\begin{tabular}{p{3.6 in} p{3.0 in}}
\hangindent=2em \texttt{hashdb <command> <options> <args>} & \hangindent=2em Run hashdb command, \texttt{-q} for quiet mode, \texttt{-f <flags>} to control B-Tree flags \\
\end{tabular}
\\
\\
\textbf{New Database} \\
\begin{tabular}{p{3.6 in} p{3.0 in}}
\hangindent=2em \texttt{create [-p <hash block size>] [-m <maximum duplicates>] [<bloom settings>] <hashdb.hdb>} &
Create a new hash database \\
\end{tabular}
\\
\\
\textbf{Import/Export} \\
\begin{tabular}{p{3.6 in} p{3.0 in}}
\hangindent=2em\texttt{import [-r <repository name>] <hashdb.hdb> <dfxml.xml>} &
Import from DFXML file into hash database \\
\hangindent=2em\texttt{import\_tab [-r <repository name>] [-s <sector size>] <hashdb.hdb> <dfxml.xml>} &
Import from tab file into hash database \\
\texttt{export <hashdb.hdb> <dfxml.xml>} &
Export hash database to DFXML file \\
\end{tabular}
\\
\\
\textbf{Database Manipulation} \\
\begin{tabular}{p{3.6 in} p{3.0 in}}
\texttt{add <A.hdb> <B.hdb>} & $A + B \rightarrow B$ Add A into B \\
\texttt{add\_multiple <A.hdb> <B.hdb> <C.hdb>} & $A + B \rightarrow C$ add A and B into C\\
\hangindent=2em\texttt{add\_repository <A.hdb> <B.hdb> <repository name>} & \hangindent=2em$A + B \rightarrow B$ Add A into B but only when the repository name matches \\
\texttt{intersect <A.hdb> <B.hdb> <C.hdb>} & $A \cap B \rightarrow C$ add intersection of A and B into C\\
\texttt{intersect\_hash <A.hdb> <B.hdb> <C.hdb>} & $A \cap B \rightarrow C$ intersect into C when hashes match\\
\texttt{subtract <A.hdb> <B.hdb> <C.hdb>} & $A - B \rightarrow C$ add A but not B into C\\
\texttt{subtract\_hash <A.hdb> <B.hdb> <C.hdb>} & $A - B \rightarrow C$ add A but not hashes in B into C\\
\texttt{deduplicate <A.hdb> <B.hdb>} & Copy $A \rightarrow B$ except for hashes with duplicates \\
\end{tabular}
\\
\\
\textbf{Scan Services} \\
\begin{tabular}{p{3.6 in} p{3.0 in}}
\texttt{scan <path or socket> <dfxml.xml>} & Scan DFXML file for matching hashes \\
\texttt{scan\_hash <path or socket> <hash value>} & Scan for hash match \\
\texttt{scan\_expanded [-m <number>] <hashdb.hdb> <dfxml.xml>} & Scan DFXML file for matches showing all sources\\
\texttt{scan\_expanded\_hash [-m <number>] <hashdb.hdb> <hash value>} & Scan for hash match showing all sources\\
\texttt{server <hashdb.hdb> <port number>} & Start scan service at port\\
\end{tabular}
\\
\\
\textbf{Statistics}\\
\begin{tabular}{p{3.6 in} p{3.0 in}}
\texttt{size <hashdb.hdb>} & Print sizes of internal database tables \\
\texttt{sources <hashdb.hdb>} & Print source metadata \\
\texttt{histogram <hashdb.hdb>} & Print hash distribution \\
\texttt{duplicates <hashdb.hdb> <number>} & Print hashes sourced the given number of times \\
\texttt{hash\_table <hashdb.hdb> <source\_id>} & Print the hashes associated with this source index\\
\hangindent=2em\texttt{expand\_identified\_blocks [-m <number>] <hashdb.hdb> <identified\_blocks.txt>} & \hangindent=2em Expand to include source information for each source \\
\hangindent=2em\texttt{explain\_identified\_blocks [-m <number>] <hashdb.hdb> <identified\_blocks.txt>} & \hangindent=2em Print information about less frequently observed hashes\\
\end{tabular}
\\
\\
\textbf{Tuning}\\
\begin{tabular}{p{3.6 in} p{3.0 in}}
\hangindent=2em\texttt{rebuild\_bloom [<bloom settings>] <hashdb.hdb>} & Rebuild Bloom filter \\
\hangindent=2em\texttt{upgrade <hashdb.hdb>} & Make database from v1.0.0 compatible with v1.1.0\\
\end{tabular}
\\
\\
\textbf{Performance Analysis}\\
\begin{tabular}{p{3.6 in} p{4 in}}
\hangindent=2em\texttt{add\_random [-r <repository name>] <hashdb.hdb> <count>} & Add random hashes, log performance in \texttt{log.xml}\\
\texttt{scan\_random <hashdb.hdb>} & Scan random hashes, log performance in \texttt{log.xml}\\
\end{tabular}
\\
\\
\textbf{\bulk Scanner}\\
%\begin{tabular}{p{5.8 in} l}
\begin{tabular}{p{5.6 in} p{2 in}}
\texttt{bulk\_extractor -E hashdb -S hashdb\_mode=import -o outdir1 -R my\_import\_dir} & Import directory\\
\hangindent=2em\texttt{bulk\_extractor -E hashdb -S hashdb\_mode=scan -S hashdb\_scan\_path\_or\_socket= outdir1/hashdb.hdb -o outdir2 my\_image2} & Scan image\\
\end{tabular}
\end{footnotesize}
\end{small}

