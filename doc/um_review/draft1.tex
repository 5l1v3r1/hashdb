% To build, please type: pdflatex draft1.tex

\documentclass[12pt,twoside]{article}
\usepackage[bf,small]{caption}
\usepackage[letterpaper,hmargin=1in,vmargin=1in]{geometry}
\usepackage{paralist} % comapctitem, compactdesc, compactenum
\usepackage{titlesec}
\usepackage{titletoc}
\usepackage{times}
\usepackage{hyperref}
\usepackage{algorithmic}
\usepackage{graphicx}
\graphicspath{{./graphics/}}
\usepackage{xspace}
\usepackage{verbatim}
\usepackage{url}
\usepackage{float}
\hyphenation{Sub-Bytes Shift-Rows Mix-Col-umns Add-Round-Key}

\setlength{\parskip}{12pt}
\setlength{\parindent}{0pt}

\newcommand{\hdb}{\emph{hashdb}\xspace}
\newcommand{\libhdb}{\emph{libhashdb}\xspace}
\newcommand{\bulk}{\emph{bulk\_extractor}\xspace}
\newcommand{\hashid}{\emph{hashid}\xspace}
\newcommand{\hid}{\emph{hashid}\xspace}
\newcommand{\mdd}{\emph{md5deep}\xspace}
\newcommand{\bev}{\emph{Bulk Extractor Viewer}\xspace}

\begin{document}

Thanks for the draft, it provides a great basis for showing what I need to
explain.

\subsubsection*{One-page quickstart}

If nothing is imported, hashdb will indicate that nothing was imported.
I would prefer to have users feel they do not need to go down
into the \hdb directory to look for a 0-length file.
Import and delete actions provide specific statistics as to what happened
during an import request.
They should consult this.
It is printed when the command completes
and is logged in the \texttt{log.xml} file.

Here are the change statistics tracked.
Their value is the number of times the event happened during the command.

\begin{verbatim}
  uint32_t hashes_inserted;
  uint32_t hashes_not_inserted_wrong_hash_block_size;
  uint32_t hashes_not_inserted_file_offset_not_aligned;
  uint32_t hashes_not_inserted_wrong_hashdigest_type;
  uint32_t hashes_not_inserted_exceeds_max_duplicates;
  uint32_t hashes_not_inserted_duplicate_element;

  uint32_t hashes_removed;
  uint32_t hashes_not_removed_wrong_hash_block_size;
  uint32_t hashes_not_removed_file_offset_not_aligned;
  uint32_t hashes_not_removed_wrong_hashdigest_type;
  uint32_t hashes_not_removed_no_hash;
  uint32_t hashes_not_removed_no_element;
\end{verbatim}

Meaning:
\begin{compactitem}
\item Number of hashes that were inserted.
\item Number of hashes that were not inserted
becasue the hash block size of the block requested for insert was
incorrect.
For example if \hdb was set to require a hash block size of 4096,
and the file size is 5096 bytes,
the last block hash size will be 100 bytes, which is not valid,
so it will not be inserted.
\item Number of hashes not inserted because the file offset was not aligned.
If the database expects block size of 4096 and the \hdb or \libhdb user
tries to add a hash at block 512, \hdb will detect that 512 does not fall on
a 4096 byte boundary $(512 \% 4096 \ne 0)$.
This happens if the user mistakenly provides unworkable parameters.
\item number of hashes with the wrong hashdigest type:
The \hdb is configured to accept MD5
but the user attempts to import 500 SHA1 hashes.
\item exceeds max duplicates: The user sets max duplicates with \texttt{-m 20}
and the run attempts to import 30 hashdigests calculated from 30 NULL blocks
of input, so we see 20 max duplicates.
\item duplicate element: The user attemts to import a hash
where the hash value, repository name, filename, and its file offset
are all the same.
\item hashes not removed, no hash: The user attempts to remove 20
hashes but these 20 hash values are not in the database.
\item hashes not removed, no element: the user attempts to remove 20 hashes
specifically identified by 
hash value, repository name, filename, and its file offset.
It is likely that the user already deleted these or never added them
in the first place, indicating a possible mistake in database management.
\end{compactitem}

After each command to change a database, values $> 0$ are written
in the \texttt{log.xml} file and to \texttt{stdout}.

\subsubsection*{Contents of the \hdb Database (not documented yet)}
The \hdb database contains these files:

\begin{verbatim}
bloom_filter_1
bloom_filter_2
hash_duplicates_store
hash_store
history.xml
log.xml
settings.xml
source_filename_store.dat
source_filename_store.idx1
source_filename_store.idx2
source_lookup_store.dat
source_lookup_store.idx1
source_lookup_store.idx2
source_repository_name_store.dat
source_repository_name_store.idx1
source_repository_name_store.idx2
\end{verbatim}

These files are:
\begin{compactitem}
\item \texttt{.xml} files containing configuration settings and logs.
\item up to 2 bloom filter files used for improving the speed of hash lookups.
\item binary files containing stored hashes and hash duplicates
from multiple sources.
\item binary files that allow lookup of hash source names.
\end{compactitem}
Of these, the following may be interesting to the user:
\begin{compactitem}
\item \texttt{log.xml} \\
Every time a command is run that changes the content of the database,
this file is replaced with a log of the run.
The log includes
the command name, information about \hdb including the command typed
and how \hdb was compiled,
information about the operating system \hdb was just run on,
a copy of the \hdb settings,
timestamps indicating how much time the command took,
and the specific \hdb changes applied, described above.
\item \texttt{settings.xml} \\
Includes the build environment and execution environment,
described above,
and the settings the user requested when the \hdb was created,
see \hdb settings and bloom filter settings options.
\item \texttt{history.xml} \\
The purpose of this file is to retain full attribution for a database.
On every change, \texttt{log.xml} is appended.
On every change applied from another database
(or from two databases, as is the case with the \texttt{add\_multiple} command),
the history of that database or databases is appended also.
It's messy and hard to follow, but it does provide full attribution.
\end{compactitem}

\subsubsection*{Introduction, Overview of \hdb}
Please replace ``pornography'' with ``child exploitation''
to map more accurately to perceived actionable concerns.

Create hash databases of SHA256 cryptographic \textbf{block} hashes,
as opposed to file hashes.

The \texttt{md5deep}, \texttt{sha1deep} etc.\ toolset
and the \hdb \texttt{export} command
are the only tools that generate DFXML output containing block hashes.

\texttt{tcpflow} does not, but \texttt{tcpflow} does provide
great usage describable elsewhere: it converts Ethernet traffic
into files so that their block hashes may be calculated
and stored in the block hash database.
Then someone can use the \hdb libray API and scan Ethernet traffic
against hashes in the database populated by hashes
from blacklist traffic values.

Provide source information for hash values.
Typically, hash values are sourced from files or directories of files
using \texttt{md5deep} with the recursive directory \texttt{-r} option.
If hash values are sourced from raw media using the \bulk \hid scanner
in \texttt{import} mode, then the Forensic Path is used
as the source information.

Regarding paper "Distinct Sector Hashes for Target File Identification":

Adding one byte to the beginning of a file stored on disk will
totally change block hash values as well,
so this argument is wrong.

The hashcode for a file is the same no matter what the system is on,
so that assertion is wrong.

The point is that the thinking and the capability set
when managing data on the block level instead of the file level is different.

File hashes: the file must be complete to generate a file hash.
A file carver can pull together a file and generate a valid file hash.
We think in terms of artifacts being files.
We like the media being examined to be a filesystem
so we can navigate through directories of files.

Block hashes:
We manage blocks of data.
We don't carve data into files.
We don't need a filesystem or files.
Artifacts are at the block scale (usually 4096 bytes), not at the file scale.
Artifacts must be sector-aligned for \hdb to find hashes.
\hdb works well with hard disks and operating systems that fragment
data into discontiguous blocks yet still sector-align media
because scans are performed along sector boundaries.

By working at the block resolution, we can find part of a file when
the rest of the file is missing, such as with a giant video file
where only part of the video is on disk.
We can analyze network traffic.
We can find artifacts that are sub-file, for example embedded content
in a \texttt{.pdf} document.

Please use ``block hashing'', not ``sector hashing''.
A disk drive may have 512-byte sectors,
but we may want to hash along 4096-byte blocks.

When importing block hashes, they are block aligned.
When scanning for block hashes,
we calculate block hashes along every sector boundary.

Block hashing does not provide any special encryption capability, as indicated.
\hdb does not decrypt.
Encrypted blocks may be hashed
and this block hash value may be stored in the database
so that when searching, we may find a block hash that matches.

Cryptographic hashes are calculated from hash blocks
\textbf{and are stored} along with source information...

\subsubsection*{How \hdb Works}
Again: DFXML containing block hashes currently is only available
from the \texttt{md5deep} toolset or \hdb \texttt{export}.

A \hdb block hash database may be populated these ways:
\begin{compactitem}
\item Importing from DFXML.
\item Copying from another \hdb database.
\item Importing from \bulk \hid scanner in \texttt{import} mode.
\end{compactitem}

Please change to: hash databases can be exported,
\textbf{added to and subtracted from}, and shared.
Merge is one set operator among add, subtract, and intersect.

Also, somewhere, please indicate that hashes in databases can be treated
as sets, as supported by the add, subtract, and intersect commands.

Somewhere, please indicate that the source of a block hash has three parts:
\begin{compactitem}
\item Repository Name:  When the user imports from a DFXML file
generated by \texttt{md5deep},
there is no repository name attributed to the import effort.
This is where the user uses the \texttt{import -r <repository name>} option.
We intend the repository name to indicate something about the dataset,
such as ``Company X's intellectual property files''.

If a repository name is not provided, the DFXML filename itself is used.

When \hdb exports databases, for each block hash,
the associated repository name as well as the associated filename and offset
are exported in the DFXML content.

\item Filename: The file from which the block hash was sourced.
\item File Offset: the offset, in bytes into the file,
where the block hash was calculated.
\end{compactitem}



\subsubsection*{2.2 DFXML}
Listing 1 illustrates nothing because it has no block hash values.
In fact, \texttt{tcpflow} cannot be used here
because it can't generate block hashes; it is only able to 
generate MD5 hashes of complete packets.

The only workflow that could use \texttt{tcpflow}
would be to use \texttt{tcpflow} to save captured traffic
to files and then to use \texttt{md5deep}
to generate block hashes from files.

Here is an excerpt of a usable DFXML file generated by \texttt{md5deep}
from the demo file at
\url{http://digitalcorpora.org/downloads/hashdb/demo/demo\_video\_dfxml}:

\begin{verbatim}
  <fileobject>
    <filename>/home/bdallen/demo/demo_video.mp4</filename>
    <filesize>10630146</filesize>
    <ctime>2014-01-30T20:20:39Z</ctime>
    <mtime>2014-01-30T19:04:59Z</mtime>
    <atime>2014-01-30T20:04:52Z</atime>
    <byte_run file_offset='0' len='4096'>   
      <hashdigest type='MD5'>63641a3c008a3d26a192c778dd088868</hashdigest>
    </byte_run>
    <byte_run file_offset='4096' len='4096'>   
      <hashdigest type='MD5'>c7dd2354e223c10856469e27686b8c6b</hashdigest>
    </byte_run>
    <byte_run file_offset='8192' len='4096'>   
      <hashdigest type='MD5'>ff540fda05d008ccebf2cca2ec71571d</hashdigest>
    </byte_run>
    <byte_run file_offset='12288' len='4096'>   
      <hashdigest type='MD5'>d3de47d704e85e0f61a91876236593d3</hashdigest>

...

    <byte_run file_offset='10625024' len='4096'>   
      <hashdigest type='MD5'>d2d958b44c481cc41b0121b3b4afae85</hashdigest>
    </byte_run>
    <byte_run file_offset='10629120' len='1026'>   
      <hashdigest type='MD5'>4640564a8655d3b201a85b4a76411b00</hashdigest>
    </byte_run>
    <hashdigest type='MD5'>a003483521c181d26e66dc09740e939d</hashdigest>
  </fileobject>
\end{verbatim}

The interesting stuff is in the \texttt{byte\_run} tag.
The \texttt{file\_offset} attribute is the number of bytes into the file
where the cryptographic block hash was calculated.

The \texttt{len} attribute indicates the size of the block.
Notice the last length hashed, 1026.
When importing hashes from 4KiB blocks, this last entry will be rejected
because $1026 \ne 4096$.
The \texttt{hashes\_not\_inserted\_wrong\_hash\_block\_size} change variable 
will be 1 because of this entry.

The \texttt{hashdigest} tag identifies that hash algorithm (MD5)
and the long hexadecimal hash value.
Note that the length, in characters for the hexadecimal value
for MD5, SHA1, and SHA256 is 16, 20, and 32, respectively.

The filename to attribute these hashes to is specified in the \texttt{filename}
tag.

Note that the DFXML that \hdb exports is compatible
but different from \texttt{md5deep}.
Here is an example excerpt, available at
\url{http://digitalcorpora.org/downloads/hashdb/demo/demo\_video\_exported\_dfxml}:

\begin{verbatim}
  <fileobject>
    <repository_name>repository_demo_video_dfxml</repository_name>
    <filename>/home/bdallen/demo/demo_video.mp4</filename>
    <byte_run file_offset='6938624' len='4096'>
      <hashdigest type='MD5'>0016aa775765eb7929ec06dea25b6f0e</hashdigest>
    </byte_run>
  </fileobject>
  <fileobject>
    <repository_name>repository_demo_video_dfxml</repository_name>
    <filename>/home/bdallen/demo/demo_video.mp4</filename>
    <byte_run file_offset='3837952' len='4096'>
      <hashdigest type='MD5'>00183a37c80b3ee02cb4bdd3e7d7e9d2</hashdigest>
    </byte_run>
  </fileobject>
  <fileobject>
    <repository_name>repository_demo_video_dfxml</repository_name>
    <filename>/home/bdallen/demo/demo_video.mp4</filename>
    <byte_run file_offset='5652480' len='4096'>
      <hashdigest type='MD5'>00513c9484ebc957eb928adf30504bc9</hashdigest>
    </byte_run>
  </fileobject>
\end{verbatim}
Differences:
\begin{compactenum}
\item The first offset is 6938624, not 0,
because the output is sorted by hash value. 
\item There is a \texttt{fileobject} tag wrapping every individual hash.
\item Every entry includes a \texttt{repository\_name} tag.
\end{compactenum}

\subsubsection*{2.3 Using the Hash Databases}
No, the \hdb library API now only provides import and scan capabilities
as seen and used by the \hdb \hid scanner.
For import:
\begin{compactitem}
\item Upon instantiation, a new database is created at the specified directory.
\item Arrays of Hashes are imported, along with their source information.
\item Upon destruction, the new database is closed, and the changes
are written in the new \texttt{log.xml} file.
\end{compactitem}

For scanning:
\begin{compactitem}
\item Upon instantiation, an existing database is opened for reading
at the specified path or socket.
\item Scans may be performed.
\item Upon destruction, resources are closed.
Since we are scanning, there are no changes to log.
\end{compactitem}

Please include the listing of the \texttt{hashdb.hpp} header file
as an appendix.
A copy of this file is available at
\url{http://digitalcorpora.org/downloads/hashdb/hashdb.hpp}.

Please also include a copy of the \hdb \hid scanner options in the appendix.
This is simply an excerpt from \hdb usage,
"Using the bulk\_extractor hashid scanner".
This makes three appendix listings: \hdb help, the \hdb library API file,
and the \hdb scanner options.

\subsubsection*{3.1.1 Installing on Linux or Mac}
Please spell out un-tar: Right-click "Extract To..." on the file
or type \texttt{tar -xvf hashdb-<x.y.z>.tar.gz}.

Also, please indicate that sudo is not required, as described under
'If you do not wish to use sudo' at
\url{https://github.com/simsong/hashdb/wiki/Installing-hashdb}.

Please also add this fact:
When building \bulk, the \hid scanner will naturally be built
but only if the \hdb library has been installed.
Otherwise, \bulk will build without the \hid scanner.
To check that the \hid scanner is enabled,
observe that it is enabled in the output of running \texttt{./configure}
or type \texttt{bulk\_extractor -h} and look for \hid setting options.

For the next available version of \bulk,
please use v1.4.5, not v1.4.1.
The v1.4.1 that I put online at
\url{http://digitalcorpora.org/downloads/hashdb/}
was hacked together for demo purposes,
and is not formally available yet.

\subsubsection*{Installing on Windows}
Please describe the warning as a Windows safeguard,
not as an antivirus measure.
The installer should never be flagged as a virus.

Please have the user select only one configuration,
the 64-bit configuration, unless it is incompatible with their old 32-bit
Windows system.
This goes for the \bulk Windows installation, too.

\subsubsection*{3.2 Run \hdb tools}
TBD.

\end{document}

