% To build, please type: pdflatex draft1.tex

\documentclass[12pt,twoside]{article}
\usepackage[bf,small]{caption}
\usepackage[letterpaper,hmargin=1in,vmargin=1in]{geometry}
\usepackage{paralist} % comapctitem, compactdesc, compactenum
\usepackage{titlesec}
\usepackage{titletoc}
\usepackage{times}
\usepackage{hyperref}
\usepackage{algorithmic}
\usepackage{graphicx}
\graphicspath{{./graphics/}}
\usepackage{xspace}
\usepackage{verbatim}
\usepackage{url}
\usepackage{float}
\hyphenation{Sub-Bytes Shift-Rows Mix-Col-umns Add-Round-Key}

\setlength{\parskip}{12pt}
\setlength{\parindent}{0pt}

\newcommand{\hdb}{\emph{hashdb}\xspace}
\newcommand{\libhdb}{\emph{libhashdb}\xspace}
\newcommand{\bulk}{\emph{bulk\_extractor}\xspace}
\newcommand{\hashid}{\emph{hashid}\xspace}
\newcommand{\hid}{\emph{hashid}\xspace}
\newcommand{\mdd}{\emph{md5deep}\xspace}
\newcommand{\bev}{\emph{Bulk Extractor Viewer}\xspace}
\newcommand{\fiwalk}{\emph{fiwalk}\xspace}

\begin{document}
Draft review 3b: review of completeness and flow.

Section 3.2.1: Creating a DFXML file:
Please add motivation for \texttt{md5deep -r} option.
We expect users to typically create block hases from thousands of files
recursively in directories,
and not from single files as shown in these nice unencumbered examples
that are easier to read.

Should we have a table for the Bloom Filter settings?
These settings may be applied when a database is created,
or they may be applied later to tune scan performance
using the \texttt{rebuild\_bloom} command.
Maybe a forward reference to the Bloom Filter section from these two places?

The repository name is an important concept for \hdb because
it keeps sources straight when hashes from many sources
are added together and filenames are not unique.
I may add two separate but similar databases
with partial overlap, resulting in some duplicate hashes
from multiple sources with the same filename.
I need the repository name to know what import to track back to.
The \texttt{import} command is the only command where
inputting a repository name is an option.
By default, the repository name used is the filename.
When exporting, the repository name is exported.
When importing from DFXML created by \hdb,
the optional repository name doesn't get used
because the \hdb export command defines all repository names,
leaving no room for choosing a default or optional value.
This information might fit at 3.2.2.

Section 3.2.2 is large and might need split.
Creating a database and populating it are separate steps.

Section 4: Use Cases:
I began reading this section thinking 'what can \hdb do for me'.
Section 4.1 presents how I can manage databases,
but I am expecting to see what I can do with \hdb
before wondering how to manage databases to do those things.

Please add information about commands so the effect of each is clear:
\begin{compactitem}
\item Commands that add or import hashes can result in hash duplicates
but source inormation will be unique.
If hash and source values are identical (including repository name)
nothing will be added.
Always please inspect statistics (Table 3) to be sure of what really happened.
\item Import has the \texttt{repository\_name} option.
It's really important that users get started on the right foot with this
or they will end up with vague names and wonder where the sources
in their database really came from.
\item While the import and add commands add multiple hashes,
tracking source information,
the intersect and subtract commands do not match source information to act.
An intersection is when hashes match even when source information does not.
Hash values are subtracted out even when source information does not match.
\item We provide two philosophies for mitigating duplicate hash bloat:
  \begin{compactitem}
  \item If we know we haven't imported the same blacklist data twice,
  and we don't want to manage a 'whitelist' database,
  \texttt{deduplicate} is a quick and easy way to get rid of low-entropy noise.
  \item If our database has blacklist data from more than one source
  or we just want tighter control about what we want to remove
  and we are willing to use a 'whitelist' database
  to remove hashes to improve lookup speed
  or to reduce noise about uninteresting hashes found,
  use \texttt{subtract}.
  \end{compactitem}
\end{compactitem}






\end{document}

